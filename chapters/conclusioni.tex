\chapter{Conclusioni}


In questa relazione è stato descritto il lavoro svolto durante l'attività di
tirocinio, inizialmente presentando la necessità di una piattaforma web che permetta
di raccogliere e rendere disponibili in un unico canale le risorse linguistiche
di SapienzaNLP, per poi ripercorrere le principali fasi e scelte affrontate.

Prima dello sviluppo di questo tirocinio era possibile che le modalità di accesso
alle risorse fossero molto diverse diverse tra loro e questo comportava ovviamente
delle difficoltà anche da un punto di vista di gestione. L'utilizzo di queste
è inoltre sempre vincolato dal rispetto delle relative licenze, che solitamente
venivano semplicemente fornite allegate alla risorse scaricate.

Tra gli aspetti che sono stati migliorati troviamo quindi la possibilità di accedere,
tramite un'unica coppia di username e password, a tutte le risorse pubblicate.
La stessa piattaforma e la stessa modalità di autenticazione include inoltre
anche la parte di gestione di quelle disponibili e di approvazione delle
richieste effetuate dagli utenti. In questo modo è stata eliminata la dispersività
che era presente in entrambi i casi ed è stato possibile definire delle modalità
di accesso condivise per tutte le risorse, così da garantire anche una maggiore
flessibilità in caso di eventuale necessità di modifiche.

Un altro miglioramento rispetto alla precedente organizzazione è relativo alla
generazione dinamica di un modulo pdf per la richiesta di accesso alle risorse.
Da un punto di vista di fruibilità si ha il vantaggio di avere i dati dell'utente
inseriti automaticamente, a differenza del classico approccio in cui il modulo è
anonimo e richiede la compilazione nonostante il sistema che lo fornisce conosca
già i dati necessari.
Il modulo risolve inoltre la gestione delle licenze, in quanto quelle relative
alle risorse richieste saranno incluse nel pdf, con relativa clausola specifica
per approvarle esplicitamente firmando.

L'aver realizzato la piattaforma con un approccio modulare ed estendibile è
infine da considerare come il principale vantaggio. È stata infatti ridotta al
minimo la dipendenza tra i vari livelli dell'architettura, così da rendere
possibile facilmente future modifiche e nuove implementazioni. Il fatto che i dati
vengano messi a disposizione tramite chiamate API lascia massima libertà sulla
scelta delle tecnologie per lo sviluppo dell'interfaccia utente, consentendo ad
esempio la creazione di un'equivalente applicazione mobile con le stesse
funzionalità della versione web.
Allo stesso modo la scelta di una nuova tecnologia per il database o per la gestione
della logica applicativa comporterà sempre la sola implementazione del nuovo
modulo.

\paragraph{}
Durante il tirocinio ho avuto modo di applicare alcuni dei concetti appresi nei
vari corsi di studio, dalla gestione delle varie fasi di progettazione
all'applicazione di buone tecniche e paradigmi di programmazione. Mi ha dato
inoltre l’opportunità di imparare e approfondire alcune tecnologie e approcci,
migliorando la mia conoscenza ed esperienza di progettazione e sviluppo per il web.



\section{Sviluppi futuri}
Essendosi conclusa insieme al tirocinio anche la fase più sostanziosa della
realizzazione della piattaforma, è importante valutare a questo punto quelli che
potrebbero essere degli sviluppi futuri per migliorare la qualità e l'utilità del
lavoro svolto.

\paragraph{Integrare un'area personale per gli utenti}
La modalità di accesso alle risorse definita attualmente per la piattaforma prevede
che l'utente riceva tramite email i link per il download, limitando l'utilizzo
dell'interfaccia alla sola fase di richiesta.
Implementando un'area personale, si potrebbe mostrare uno storico delle richieste
effettuate, e grazie ad esso risolvere l'eventualità in cui se ne effettua
più di una per le stesse risorse, ad esempio nel caso in cui non si è più in
possesso dei link di download ricevuti tramite email. Inoltre potrebbe permettere
la modifica dei dati inseriti in fase di registrazione, attualmente possibile solo
tramite il pannello di amministrazione.

\paragraph{Anonimizzare gli utenti disabilitati}
La scelta di disabilitare gli utenti piuttosto che eliminarli è stata fatta per
garantire consistenza nei dati presenti nel database e mantenere tutte le attività
svolte dall'utente nonostante non possa più utilizzare la piattaforma.
Questo però preclude la possibilità di cancellare i dati dell'utente (a meno di
azione diretta sul database) nel caso si renda necessario ad esempio per esplicita
richiesta. Si potrebbe pensare ad un sistema di anonimizzazione dell'utente da
cancellare, applicabile anche automaticamente dopo un lasso di tempo definito dalla
disabilitazione dell'account.

\paragraph{Possibilità di apporre Firma Digitale per i moduli pdf}
Una funzionalità che potrebbe migliorare la fruibilità della procedura di richiesta
per l'accesso alle risorse e allo stesso tempo far ottenere maggiori garanzie sul
modulo firmato è l'integrazione della doppia possibilità di firmare per via autografa
o per via di Firma Digitale.

La firma elettronica qualificata (FEQ) -- o digitale -- sostituisce una tradizionale
firma apposta su carta, garantendo anche autenticità, integrità e validità legale
dei documenti, poiché un titolare di FEQ può firmare digitalmente i propri documenti
solo tramite un certificato digitale di sottoscrizione, rilasciato da enti autorizzati.

\paragraph{Possibilità di autenticarsi tramite Identità Digitale}
Un'altra integrazione da valutare è la possibilità di effettuare l'autenticazione
utilizzando un sistema di identificazione digitale verificato.
Prendiamo come esempio \textbf{SPID}\footnote{\url{https://www.spid.gov.it}},
il Sistema Pubblico di Identità Digitale implementato in Italia, che consente di
accedere ai servizi della pubblica amministrazione e dei privati aderenti tramite
un'identità digitale unica.

La possibilità di autenticarsi utilizzando SPID come sistema di riconoscimento
non avrebbe l’obiettivo di eliminare tutta la fase di registrazione
(v. \autoref{subsec:authentication}), ma di offrire un’alternativa all’utente che
permetta l'autocompilazione dei dati anagrafici e l'accettazione dei termini di
licenza per le risorse richieste senza necessità di firmare un modulo pdf.

\paragraph{Valutare tecnologie alternative}
Essendo stato lo sviluppo e il testing della piattaforma svolto solamente in locale,
in seguito al rilascio ufficiale si potrebbero effettuare delle nuove valutazioni
sulle tecnologie utilizzate e sulle scelte effettuate. Ad esempio sulla tipologia
di database, piuttosto che sul framework per lo sviluppo dell'Application tier.
Ci possono essere molteplici motivi per cui un cambiamento di questo tipo
potrebbe rendersi necessario, ad esempio per un discorso di performance o di
consumi. Importante è quindi aver progettato il sistema per adattarsi facilmente
anche a modifiche che comporterebbero la sostituzione di un intero livello
architetturale (v. \autoref{sec:architettura}).