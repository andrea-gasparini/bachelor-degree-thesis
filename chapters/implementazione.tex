% !TeX root = ../relazione.tex

\chapter{Implementazione}


\section{Data tier}
Questo modulo comprende il database, per il quale è stato scelto \textbf{MariaDB},
una soluzione open source sviluppata dagli autori di MySQL, e implementa delle
API per poter stabilire facilmente una connessione con esso ed effettuare
modifiche e interrogazioni, mappando le operazioni \textbf{CRUD} (Create, Retrieve,
Update e Delete).

\paragraph{Pattern DAO}
Per l'implementazione si è utilizzato \textbf{Java} come linguaggio di programmazione
e il \textbf{Design Pattern DAO} (Data Access Object). Per ogni entità del database
sono state quindi definite:
\begin{itemize}
	\item Una classe \textbf{DTO} (Data Transfer Object), che rappresenta la
	relativa entità, le cui istanze vengono utilizzate per lo scambio dei dati
	con il database
	\item Un'interfaccia \textbf{DAO} che definisce i metodi che si possono
	utilizzare per interagire con la relativa entità del database, tra cui le
	operazioni CRUD necessarie per assolvere i requisiti del sistema
	\item Almeno una classe che implementi il DAO, responsabile dell'effettiva
	connessione con il database e della logica con cui i metodi accedono ai dati
\end{itemize}

\lstinputlisting[language=Java, caption=Interfaccia DAO relativa all'entità \textit{user}]{assets/code/UserDAO.java}

Questa struttura migliora ulteriormente la modularità e la flessibilità del
progetto (\autoref{sec:architettura}); ad esempio se fosse necessario utilizzare
una diversa tecnologia per il database sarebbe sufficiente una nuova
implementazione del DAO, che ovviamente manterrebbe gli stessi metodi.

\paragraph{Project Lombok}
Una libreria open source che è stata molto utilizzata per ridurre i tempi di
scrittura e migliorare la leggibilità del codice è \textbf{Project Lombok};
un \textit{Annotation processor} che, tramite delle annotazioni nel codice,
sostituisce la scrittura di metodi molto comuni e simili tra loro (come
\textit{setter} o \textit{getter}).
Nelle classi che rappresentano dei dati, come le classi DTO, questo è stato
essenziale per poter scrivere in poche righe un codice chiaro e funzionale;
ad esempio, la classe \texttt{User} (\ref{code:user-dto}), grazie alle annotazioni
\texttt{@Getter}, \texttt{@Builder} e \texttt{@Setter}, dispone di metodi get
per ogni campo, di un'implementazione del pattern Builder e di un metodo set per
il campo \texttt{id}.

\lstinputlisting[language=Java, label={code:user-dto}, caption=Classe DTO relativa all'entità \textit{user}]{assets/code/UserDTO.java}



\section{Application tier}
In questo modulo viene gestita la parte logica ed applicativa della piattaforma,
tra cui il controllo della sessione utente e i vincoli derivanti dai requisiti;
come ad esempio la possibilità di accedere ad alcune funzionalità solo per certe
tipologie di utenti o il controllo per cui i link di download siano validi solo
per le richieste approvate e solo per l'utente che ha effettuato la richiesta. 


\subsection{Il framework Spring}
Il linguaggio utilizzato è sempre \textbf{Java} ma accoppiato a \textbf{Spring},
un framework per lo sviluppo di applicazioni web, nello specifico nella versione
\textbf{Boot} \cite{spring:boot}, che ne semplifica la configurazione e garantisce
la possibilità di avere un web server \textit{stand-alone} direttamente eseguendo
la classe principale del progetto o effettuandone la build, che produrrà un file
JAR eseguibile con la stessa funzionalità.

\paragraph{Services} \label{par:spring-services}
Per la gestione dell'invio delle email, della generazione del modulo pdf e
dell'interazione con il file system per salvatare i moduli inviati e permettere
il download delle risorse, sono state definite delle classi annotate come
\texttt{@Service}, ovvero relative solo a parte della \textit{business logic}
dell'applicazione. I Service vengono poi utilizzati nel resto del progetto
tramite la \textbf{Dependency Injection} di Spring, che si occupa di fornire la
stessa istanza a tutti i componenti che ne richiedono l'utilizzo. La forma di
injection applicata è quella tramite il costruttore della classe 
(\ref{code:constr-inj}) che viene parametrizzato automaticamente dal framework
all'avvio dell'applicazione.

\lstinputlisting[language=Java, label={code:constr-inj}, caption=Esempio di Constructor Injection per \texttt{EmailService}]{assets/code/AuthenticationController.constructor.java}

\paragraph{Controllers}
Tra i componenti principali del framework troviamo le classi Controller, definite
tramite l'annotazione \texttt{@RestController} e adibite ad eseguire le funzionalità
nel rispetto dei vincoli dell'applicazione. I metodi di queste classi vengono
mappati a degli endpoint del web server che permettono la loro invocazione tramite
richieste HTTP, creando così delle API accessibili tramite l'interfaccia utente.

Ad esempio nella classe \texttt{AuthenticationController} è stato implementato un
metodo \texttt{signUp} (\ref{code:auth-controller-signup}) che permette di
effettuare la registrazione di un nuovo utente. Questo metodo, tramite
l'annotazione \texttt{@PostMapping}, è accessibile effettuando una richiesta POST
all'endpoint \textit{/signup}.

Nel dettaglio, effettua dei controlli sulla validità dei parametri e sull'unicità
di username e indirizzo email, per poi istanziare un nuovo oggetto \texttt{User},
inviare un'email di conferma tramite il Service (\autoref{par:spring-services})
che gestisce le email, inserire l'utente nel database tramite un oggetto DAO e
infine associare un nuovo \texttt{AccessToken} all'utente per garantirgli la sessione.

\lstinputlisting[language=Java, label={code:auth-controller-signup}, caption=Metodo \texttt{signUp} della classe \texttt{AuthenticationController}]{assets/code/AuthenticationController.signup.java}

\paragraph{Filters}
La validità delle sessioni utente e l'accesso alle funzionalità riservate per
utenti amministratori o gestori, sono state gestite tramite dei pattern negli
endpoint; utilizzando ad esempio \textit{/api/admin} come prefisso per tutti
quelli su cui era necessario verificare che l'utente fosse un amministratore
(\ref{code:admin-filter}).

Definendo delle classi che estendono \texttt{OncePerRequestFilter} è stato
possibile implementare dei controlli eseguiti prima di ogni richiesta che rispetti
il pattern specificato, ritornando un codice di errore nel caso le condizioni
non vengano soddisfatte o permettendo l'esecuzione della richiesta altrimenti.

\lstinputlisting[language=Java, label={code:admin-filter}, caption=Filtro per verificare che l'utente possa accedere ad un endpoint riservato]{assets/code/AdminFilter.java}
