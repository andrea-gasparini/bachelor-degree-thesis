% !TeX root = ../relazione.tex

\chapter{Implementazione}


\section{Data tier}
Questo modulo comprende il database, per il quale è stato scelto \textbf{MariaDB},
una soluzione open source sviluppata dagli autori di MySQL, e implementa delle
API per poter stabilire facilmente una connessione con esso ed effettuare
modifiche e interrogazioni, mappando le operazioni \textbf{CRUD} (Create, Retrieve,
Update e Delete).

\paragraph{Pattern DAO}
Per l'implementazione si è utilizzato \textbf{Java} come linguaggio di programmazione
e il \textbf{Design Pattern DAO} (Data Access Object). Utilizzando questo pattern,
per ogni entità del database è necessario definire:
\begin{itemize}
	\item Una classe \textbf{DTO} (Data Transfer Object), che rappresenta la
	relativa entità, le cui istanze vengono utilizzate per lo scambio dei dati
	con il database
	\item Un'interfaccia \textbf{DAO} che definisce i metodi che si possono
	utilizzare per interagire con la relativa entità del database, tra cui le
	operazioni CRUD necessarie per assolvere i requisiti del sistema
	\item Almeno una classe che implementi il DAO, responsabile dell'effettiva
	connessione con il database e della logica con cui i metodi accedono ai dati
\end{itemize}

\lstinputlisting[language=Java, caption=Classe DAO relativa all'entità \textit{user}]{assets/code/UserDAO.java}

Questa struttura migliora ulteriormente la modularità e la flessibilità del
progetto (\autoref{sec:architettura}); ad esempio se fosse necessario utilizzare
una diversa tecnologia per il database sarebbe sufficiente una nuova
implementazione del DAO, che ovviamente manterrebbe gli stessi metodi.

\paragraph{Project Lombok}
Una libreria open source che è stata molto utilizzata per ridurre i tempi di
scrittura e migliorare la leggibilità del codice è \textbf{Project Lombok};
un \textit{Annotation processor} che, tramite delle annotazioni nel codice,
sostituisce la scrittura di metodi molto comuni e simili tra loro (come
\textit{setter} o \textit{getter}).
Nelle classi che rappresentano dei dati, come le classi DTO, questo è stato
essenziale per poter scrivere in poche righe un codice chiaro e funzionale;
ad esempio, la classe \texttt{User} (\ref{code:user-dto}), grazie alle annotazioni
\texttt{@Getter}, \texttt{@Builder} e \texttt{@Setter}, dispone di metodi get
per ogni campo, di un'implementazione del pattern Builder e di un setter per il
campo \texttt{id}.

\lstinputlisting[language=Java, label={code:user-dto}, caption=Classe DTO relativa all'entità \textit{user}]{assets/code/UserDTO.java}
