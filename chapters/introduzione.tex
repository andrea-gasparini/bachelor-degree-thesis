% !TeX root = ../relazione.tex

\chapter{Introduzione}

L'oggetto del tirocinio è la progettazione e lo sviluppo di una piattaforma web
per migliorare la gestione delle risorse linguistiche di \textbf{SapienzaNLP},
il gruppo di ricerca del dipartimento di Informatica che si occupa di multilingual
Natural Language Understanding.
Un esempio di risorsa è \textbf{BabelNet} \cite{NavigliPonzetto:12aij}, una rete
semantica multilingue consultabile online\footnote{\url{https://babelnet.org}},
per cui è possibile effettuare richiesta di scaricarne la versione offline.

Lo scopo principale di questa nuova piattaforma è quello di avere un canale unico
per richiedere l'accesso alle risorse pubblicate, gestire lo stato di approvazione
delle richieste e definire le modalità di accesso ai dati.
Gli utenti possono visualizzare tutte le risorse disponibili per il download
e selezionarne una o più per richiederne l'accesso. L'utilizzo di queste risorse
è vincolato al rispetto delle relative licenze e viene quindi fornito all'utente
anche un modulo pdf generato automaticamente con delle clausole specifiche da firmare.
La possibilità di effettuare il download viene concessa da un amministratore, che
revisiona i dati dell'utente e il modulo di richiesta, e può garantire l'accesso
alle risorse richieste (o a parte di esse) generando dei link di download accessibili
solamente all'utente richiedente.
\newline
\newline
La relazione è divisa in {\textbf{\color{red}x}} capitoli:
l'analisi dei requisiti e delle principali funzionalità della piattaforma,
la progettazione del sistema e la descrizione delle tecnologie utilizzate,
i dettagli dell sviluppo e dell'implentazione finale.