% !TeX root = ../relazione.tex

\chapter{Introduzione}

L'oggetto del tirocinio è la progettazione e lo sviluppo di una piattaforma web
per migliorare la gestione delle risorse linguistiche di \textbf{SapienzaNLP},
il gruppo di ricerca del dipartimento di Informatica che si occupa di multilingual
Natural Language Understanding.

\section{Le risorse linguistiche nell'NLP}
Il Natural Language Processing (NLP) è un settore dell'intelligenza artificiale
che studia i problemi legati a comprensione, analisi e generazione automatica del
linguaggio naturale (come ad esempio un testo scritto o una conversazione orale).
Il \textbf{Natural Language Understanding} (NLU) \cite{navigli2018natural} è un argomento
specifico dell'NLP che riguarda in particolare la capacità delle macchine di leggere
e comprendere il significato di un testo, e più in generale del linguaggio naturale.
È facile rendersi conto di quanto sia complesso l'obiettivo che l'NLU si pone,
per via della grande ambiguità che può avere il linguaggio umano, basti pensare
ai diversi significati che una stessa frase può assumere in base al contesto del
discorso.
Nell'affrontare questa tematica è quindi importante prendere in considerazione
l'analisi sintattica e semantica.

\paragraph{Cos'è una risorsa linguistica?}
Per far si che un algoritmo di NLP possa elaborare un testo basandosi su sintassi
e semantica in maniera automatica è necessario che possa "imparare" a partire
da dei dati linguistici strutturati.
Una risorsa contenente questa tipologia di dati e che ne consenta facilmente il
trattamento e l'analisi da parte di una macchina viene detta \textbf{risorsa linguistica}.
Da non confondere con i dizionari informatizzati, il cui scopo è limitato a
fornire informazioni linguistiche sul lessico di una lingua e che sono costruiti
per essere consultati solamente da un utente umano.

Inizialmente la costruzione di risorse linguistiche avveniva manualmente, come
ad esempio per WordNet e FrameNet, ma questo richiedeva un grande impegno in
termini di tempi e costi.
Solo negli ultimi anni si sono sperimentati altri approcci per la creazione
delle risorse, da quelli automatici o semi-automatici, in cui parte del lavoro
viene svolto da una macchina, a quelli collaborativi che sarebbero impensabili
senza l'ausilio di internet.

\paragraph{WordNet \cite{Miller1990}}
WordNet è una delle più importanti e storiche risorse linguistiche per l'inglese.
Il progetto, nato nel 1985 e ancora oggi molto utilizzato, si basa sull'idea
di creare un dizionario in cui la ricerca non avviene alfabeticamente ma concettualmente.
All'interno di WordNet le parole sono organizzate in gruppi di sinonimi detti
\textit{synset}, dall'inglese synonyms set, ciascuno dei quali rappresenta un
concetto lessicale.
La struttura è quella di un grafo, in cui i nodi rappresentano i synset e gli archi
le relazioni lessicali e semantico-concettuali presenti tra loro.

\paragraph{BabelNet \cite{NavigliPonzetto:12aij}}
Una delle principali risorse linguistiche pubblicate da SapienzaNLP è BabelNet,
un dizionario enciclopedico multilingue consultabile online\footnote{\url{https://babelnet.org}},
creato integrando automaticamente risorse come Wikipedia, WordNet, Wiktionary,
Wikidata e molte altre.
BabelNet è allo stesso tempo una rete semantica che collega concetti, lemmi
ed entità nominate tramite delle relazioni semantiche chiamate \textbf{Babel synset}.
Ogni Babel synset comprende diverse lingue e rappresenta uno specifico significato
raggruppando tutti i sinonimi che lo esprimono, con relative definizioni ed esempi.

Attualmente BabelNet supporta 284 lingue diverse, per un totale di 15,780,364
Babel synsets, 808,974,108 sensi, 277,036,611 relazioni semantico-lessicali.



\section{La necessità della piattaforma}
Attualmente l'utente ha diverse modalità di accesso e di approvazione per ogni
risorsa pubblicata, il che può comportare ad esempio di dover gestire diverse
credenziali di accesso (username e password). Mancano inoltre i moduli per il
consenso esplicito dell'utente all'utilizzo nel rispetto delle relative licenze.
Lo scopo è quindi quello di avere un unico portale per richiedere l'accesso alle
risorse, gestire lo stato di approvazione delle richieste e far approvare
esplicitamente le modalità con cui è concesso l'utilizzo dei dati.

Tramite la piattaforma è possibile infatti visualizzare tutte le risorse
disponibili per il download e selezionarne una o più per richiederne l'accesso.
Fornisce inoltre un modulo pdf, generato dinamicamente, contenente le clausole
specifiche da dover firmare per poter effettuare la richiesta. La possibilità di
effettuare il download può essere poi concessa da un amministratore, che revisiona
i dati dell'utente e il modulo di richiesta, e può garantire l'accesso alle risorse
richieste (o a parte di esse) generando dei link di download utilizzabili
solamente dall'utente richiedente.

\paragraph{}
La relazione è divisa in 3 parti fondamentali: l'\textbf{analisi dei requisiti},
volta a migliorare la modellazione delle funzionalità della piattaforma; la
\textbf{progettazione del sistema} e della sua architettura; ed infine i dettagli
dello sviluppo e delle \textbf{fasi di implementazione}.