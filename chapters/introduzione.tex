% !TeX root = ../relazione.tex

\chapter{Introduzione}

L'oggetto del tirocinio è la progettazione e lo sviluppo di una piattaforma web
per migliorare la gestione delle risorse linguistiche di \textbf{SapienzaNLP},
il gruppo di ricerca del dipartimento di Informatica che si occupa di multilingual
Natural Language Understanding.
Un esempio di risorsa è \textbf{BabelNet} \cite{NavigliPonzetto:12aij}, una rete
semantica multilingue consultabile online\footnote{\url{https://babelnet.org}},
per cui è possibile effettuare richiesta di scaricarne la versione offline.

Attualmente l'utente ha diverse modalità di accesso e di approvazione per ogni
risorsa pubblicata, il che può comportare ad esempio a dover gestire diverse
credenziali di accesso (username e password). Mancano inoltre i moduli per il
consenso esplicito dell'utente all'utilizzo nel rispetto delle relative licenze.
Lo scopo è quindi quello di avere un unico portale per richiedere l'accesso alle
risorse, gestire lo stato di approvazione delle richieste e far approvare
esplicitamente le modalità con cui è concesso l'utilizzo dei dati.

Tramite la piattaforma è possibile infatti visualizzare tutte le risorse
disponibili per il download e selezionarne una o più per richiederne l'accesso.
Fornisce inoltre un modulo pdf, generato dinamicamente, contenente le clausole
specifiche da dover firmare per poter effettuare la richiesta. La possibilità di
effettuare il download può essere poi concessa da un amministratore, che revisiona
i dati dell'utente e il modulo di richiesta, e può garantire l'accesso alle risorse
richieste (o a parte di esse) generando dei link di download utilizzabili
solamente dall'utente richiedente.

\paragraph{}
La relazione è divisa in 3 parti fondamentali: l'\textbf{analisi dei requisiti},
volta a migliorare la modellazione delle funzionalità della piattaforma; la
\textbf{progettazione del sistema} e della sua architettura; ed infine i dettagli
dello sviluppo e delle \textbf{fasi di implementazione}.