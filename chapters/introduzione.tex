% !TeX root = ../relazione.tex

\chapter{Introduzione}

L'oggetto del tirocinio è la progettazione e lo sviluppo di una piattaforma web
per migliorare la gestione delle risorse linguistiche di \textbf{SapienzaNLP},
il gruppo di ricerca del dipartimento di Informatica che si occupa di multilingual
Natural Language Understanding.
Un esempio di risorsa è \textbf{BabelNet} \cite{NavigliPonzetto:12aij}, una rete
semantica multilingue consultabile online\footnote{\url{https://babelnet.org}},
per cui è possibile effettuare richiesta di scaricarne la versione offline.

Il problema principale è dovuto alle diverse modalità di accesso e approvazione
di ogni risorsa, che rendono frammentata la gestione e mancano ad esempio di
consenso esplicito dell'utente all'utilizzo nel rispetto delle relative licenze.
Lo scopo di è quindi di avere un unico canale per richiedere l'accesso alle
risorse pubblicate, gestire lo stato di approvazione delle richieste e far
approvare esplicitamente le modalità con cui è concesso l'utilizzo dei dati.

Tramite la piattaforma è possibile infatti visualizzare tutte le risorse
disponibili per il download e selezionarne una o più per richiederne l'accesso.
Fornisce inoltre un modulo pdf, generato dinamicamente, contenente le clausole
specifiche da dover firmare per poter effettuare la richiesta. La possibilità di
effettuare il download può essere poi concessa da un amministratore, che revisiona
i dati dell'utente e il modulo di richiesta, e può garantire l'accesso alle risorse
richieste (o a parte di esse) generando dei link di download utilizzabili
solamente dall'utente richiedente.
\newline
\newline
La relazione è divisa in {\textbf{\color{red}x}} capitoli:
l'analisi dei requisiti e delle principali funzionalità della piattaforma,
la progettazione del sistema e la descrizione delle tecnologie utilizzate,
i dettagli dell sviluppo e dell'implentazione finale.