% !TeX root = ../relazione.tex

\chapter{Analisi dei requisiti}


\section{Requisiti della piattaforma}
Inizialmente sono stati raccolti ed analizzati i requisiti principali della
piattaforma, così da poterne definire ulteriori funzionalità e guidare le
successive fasi dello sviluppo.
Gli attori\footnote{Coloro che interagiscono con il sistema} del sistema
vengono suddivisi nelle seguenti tipologie:

\begin{itemize}
	\item \textbf{Utente web}: usufruisce delle risorse di SapienzaNLP e utilizza
	la piattaforma per richiederne l'accesso e per scaricarle
	\item \textbf{Utente gestore}: gestisce le richieste effettuate, ne
	può cambiare lo stato (ad esempio approvandole o rifiutandole) e può generare
	dei link di download tramite cui scaricare le risorse associate
\end{itemize}

\paragraph{Selezione delle risorse}
Gli utenti web possono selezionare una o più risorse tra quelle disponibili nella
piattaforma per cui richiedere l'accesso, aggiungendo una descrizione obbligatoria
per indicare l'utilizzo che se ne vuole fare. A questa azione segue la generazione
di un pdf contenente un resoconto e le informazioni legali che l'utente dovrà
obbligatoriamente approvare per poter richiedere l'accesso alle risorse selezionate.

\paragraph{Richiesta di accesso} \label{par:access-request}
Gli utenti web possono sottomettere una richiesta di accesso alle risorse, inserendo
i propri dati e il pdf precedentemente generato, firmato dove richiesto.
I dati richiesti all'utente sono i seguenti: nome, cognome, username, email,
azienda/organizzazione e paese di provenienza.

\paragraph{Autenticazione amministrazione}
Gli utenti gestori, possono accedere ad un pannello di amministrazione
effettuando l'autenticazione tramite username e password.

\paragraph{Cronologia delle richieste}
Gli utenti gestori, tramite il pannello di amministrazione, possono accedere allo
storico di tutte le richieste di accesso alle risorse effettuate, ordinate
cronologicamente.

\paragraph{Dettaglio e accettazione/rifiuto richiesta}
Gli utenti gestori, tramite il pannello di amministrazione, possono accedere ai
dettagli delle richieste effettuate, visualizzarne il modulo pdf associato e
cambiare lo stato della richiesta, approvandola o rifiutandola.
La cronologia dei cambiamenti di stato, e l'utente che li ha effettuati, viene
registrata nel sistema e mostrata insieme ai dettagli della singola richiesta.

\paragraph{Generazione di link per il download}
Gli utenti gestori, tramite il pannello di amministrazione, possono generare dei
link di download per una o più risorse relativamente ad una richiesta di accesso.
I link vengono inviati automaticamente all'indirizzo email che l'utente ha specificato
in fase di sottomissione della richiesta, tramite cui potrà scaricare le risorse.



\section{Raffinamento e funzionalità aggiuntive}
I requisiti appena elencati rappresentano le funzionalità essenziali per permettere
accesso e gestione delle risorse, ovvero lo scopo principale della piattaforma.
È però necessario tenere anche conto dei dati relativi ad utenti gestori, alle
risorse e alle loro versioni; è quindi seguita una fase di raffinamento dei
requisiti e delle funzionalità.


\subsection{Utenti}
Viene aggiunta una terza tipologia di utente, l'\textbf{Utente amministratore}
che può accedere al pannello di amministrazione tramite autenticazione e, oltre
alle richieste effettuate, può visualizzare e gestire gli utenti, le risorse e
le relative versioni. Può accedere a tutte le funzionalità della piattaforma.
Le tipologie presenti nella piattaforma sono quindi le seguenti:
\begin{itemize}
	\item Utente web
	\item Utente gestore
	\item Utente amministratore
\end{itemize}


\subsection{Autenticazione} \label{subsec:authentication}
Per semplificare il processo di inserimento dei dati da parte di un utente web
che richiede l'accesso alle risorse (v. \ref{par:access-request}) ed evitare che
per ogni nuova richiesta debbano essere re-inseriti, l'accesso alla piattaforma
viene gestito da una procedura di autenticazione per tutte le tipologie di utenti.
In questo modo se uno stesso utente effettua più volte delle richieste, i dati
necessari saranno già presenti nel sistema e sarà quindi necessario aggiungervi
solamente il modulo pdf firmato.

\paragraph{Registrazione}
Viene quindi aggiunta la possibilità di registrarsi alla piattaforma inserendo i
dati che erano previsti per le richieste di accesso alle risorse. Questa procedura
permette di creare un nuovo utente web, che quindi non avrà accesso al pannello
di amministrazione.

\paragraph{Recupero password}
Per tutte le tipologie di utente viene data la possibilità di generare una nuova
password, inviata tramite email, che diventa permanente nel momento in cui
viene utilizzata per effettuare l'accesso.

\paragraph{Autenticazione}
Ad esclusione di \textbf{Registrazione} e \textbf{Recupero password}, per utilizzare
le funzionalità della piattaforma relative alla propria tipologia di utente,
deve essere prima effettuata l'autenticazione inserendo username e password.


\subsection{Gestione utenti}
Per consentire un controllo completo della piattaforma, vengono aggiunte anche
le funzionalità relative alla gestione degli utenti, rese disponibili ai soli
utenti amministratori.

\paragraph{Lista utenti esistenti}
È possibile visualizzare una lista di tutti gli utenti registrati nella piattaforma
e avere un resoconto delle relative informazioni principali, come username e
nome completo.

\paragraph{Dettaglio utente esistente}
È possibile visualizzare e modificare i dati degli utenti, ad eccezione della
password, e visualizzare lo storico delle attività, nello specifico:
\begin{itemize}
	\item delle richieste effettuate, per gli utenti web
	\item dei cambi di stato delle richieste, per gli utenti gestori o amministratori
\end{itemize}

\paragraph{Disabilitazione utente esistente}
È possibile disabilitare un utente, negandogli la possibilità di autenticarsi e,
di conseguenza, di utilizzare qualunque funzionalità della piattaforma.
Questa azione non elimina l'utente ed è reversibile, così che venga sempre
mantenuto lo storico delle richieste e dei cambi di stato effettuati.

\paragraph{Creazione utente gestore}
È possibile creare un nuovo utente gestore inserendo gli stessi dati necessari
per la registrazione spontanea di un utente web. Il nuovo utente gestore sarà
abilitato ad accedere al pannello di amministrazione, con le sole
funzionalità previste di gestione delle richieste.


\subsection{Gestione risorse}
Anche per quanto riguarda le risorse è importante consentire le funzionalità per
la loro gestione, rendendole disponibili ai soli utenti amministratori.

\paragraph{Lista risorse esistenti}
È possibile visualizzare una lista di tutte le risorse disponibili nella
piattaforma e avere una panoramica sulle relative informazioni principali.

\paragraph{Dettaglio risorsa esistente}
È possibile visualizzare in dettaglio e modificare i dati delle risorse, inoltre
visualizzare le versioni disponibili associate ad essa.

\paragraph{Eliminazione risorsa esistente}
È possibile eliminare definitivamente dalla piattaforma una risorsa e con essa
tutte le versioni associate.

\paragraph{Creazione nuova risorsa}
È possibile aggiungere una nuova risorsa, specificando un nome, una descrizione
e la licenza con cui viene fornita.

\paragraph{Creazione e modifica versione risorsa}
È possibile aggiungere e modificare una versione di una risorsa, specificandone
il nome, la data di rilascio e il file a cui fa riferimento per il download.